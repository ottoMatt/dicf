%
% API Documentation for API Documentation
% Module etude.problems.partitioning
%
% Generated by epydoc 3.0.1
% [Wed Sep  9 11:36:00 2015]
%

%%%%%%%%%%%%%%%%%%%%%%%%%%%%%%%%%%%%%%%%%%%%%%%%%%%%%%%%%%%%%%%%%%%%%%%%%%%
%%                          Module Description                           %%
%%%%%%%%%%%%%%%%%%%%%%%%%%%%%%%%%%%%%%%%%%%%%%%%%%%%%%%%%%%%%%%%%%%%%%%%%%%

    \index{etude \textit{(package)}!etude.problems \textit{(package)}!etude.problems.partitioning \textit{(module)}|(}
\section{Module etude.problems.partitioning}

    \label{etude:problems:partitioning}

%%%%%%%%%%%%%%%%%%%%%%%%%%%%%%%%%%%%%%%%%%%%%%%%%%%%%%%%%%%%%%%%%%%%%%%%%%%
%%                               Functions                               %%
%%%%%%%%%%%%%%%%%%%%%%%%%%%%%%%%%%%%%%%%%%%%%%%%%%%%%%%%%%%%%%%%%%%%%%%%%%%

  \subsection{Functions}

    \label{etude:problems:partitioning:annotateLines}
    \index{etude \textit{(package)}!etude.problems \textit{(package)}!etude.problems.partitioning \textit{(module)}!etude.problems.partitioning.annotateLines \textit{(function)}}

    \vspace{0.5ex}

\hspace{.8\funcindent}\begin{boxedminipage}{\funcwidth}

    \raggedright \textbf{annotateLines}(\textit{f})

    \vspace{-1.5ex}

    \rule{\textwidth}{0.5\fboxrule}
\setlength{\parskip}{2ex}
    annotate lines with information used for partitioning such as axiom/tp,
    and index

\setlength{\parskip}{1ex}
    \end{boxedminipage}

    \label{etude:problems:partitioning:filterClauses}
    \index{etude \textit{(package)}!etude.problems \textit{(package)}!etude.problems.partitioning \textit{(module)}!etude.problems.partitioning.filterClauses \textit{(function)}}

    \vspace{0.5ex}

\hspace{.8\funcindent}\begin{boxedminipage}{\funcwidth}

    \raggedright \textbf{filterClauses}(\textit{l})

    \vspace{-1.5ex}

    \rule{\textwidth}{0.5\fboxrule}
\setlength{\parskip}{2ex}
    ne garde que les clauses

\setlength{\parskip}{1ex}
    \end{boxedminipage}

    \label{etude:problems:partitioning:divEquNaive}
    \index{etude \textit{(package)}!etude.problems \textit{(package)}!etude.problems.partitioning \textit{(module)}!etude.problems.partitioning.divEquNaive \textit{(function)}}

    \vspace{0.5ex}

\hspace{.8\funcindent}\begin{boxedminipage}{\funcwidth}

    \raggedright \textbf{divEquNaive}(\textit{infilename}, \textit{outfilename}, \textit{nbAgents})

    \vspace{-1.5ex}

    \rule{\textwidth}{0.5\fboxrule}
\setlength{\parskip}{2ex}
    1ere naive : divides the clauses equally between the agents and writes 
    the output in a .gro.part.N file

\setlength{\parskip}{1ex}
    \end{boxedminipage}

    \label{etude:problems:partitioning:create_a_FileSol_wit_a_TP_distribution_naiveShort}
    \index{etude \textit{(package)}!etude.problems \textit{(package)}!etude.problems.partitioning \textit{(module)}!etude.problems.partitioning.create\_a\_FileSol\_wit\_a\_TP\_distribution\_naiveShort \textit{(function)}}

    \vspace{0.5ex}

\hspace{.8\funcindent}\begin{boxedminipage}{\funcwidth}

    \raggedright \textbf{create\_a\_FileSol\_wit\_a\_TP\_distribution\_naiveShort}(\textit{file\_sol}, \textit{perc}, \textit{seuilMin}={\tt 1})

    \vspace{-1.5ex}

    \rule{\textwidth}{0.5\fboxrule}
\setlength{\parskip}{2ex}
    TP = top\_clause : args = percentage, seuil min return a new File Sol 
    object with new distribution perc =[0, 100]

    faire une variante qui garde les TP originales ?? pas RANDOM !! on 
    prend les n premiers plus courtes dans l'ordre d'apparition !

\setlength{\parskip}{1ex}
    \end{boxedminipage}

    \label{etude:problems:partitioning:create_dcf_Agent_distribution}
    \index{etude \textit{(package)}!etude.problems \textit{(package)}!etude.problems.partitioning \textit{(module)}!etude.problems.partitioning.create\_dcf\_Agent\_distribution \textit{(function)}}

    \vspace{0.5ex}

\hspace{.8\funcindent}\begin{boxedminipage}{\funcwidth}

    \raggedright \textbf{create\_dcf\_Agent\_distribution}(\textit{file\_sol}, \textit{nbAgents}, \textit{method}={\tt \texttt{'}\texttt{naive\_eq}\texttt{'}}, \textit{outpath}={\tt None})

    \vspace{-1.5ex}

    \rule{\textwidth}{0.5\fboxrule}
\setlength{\parskip}{2ex}
    on doit prendre en compte l'index des lignes désormais method = 
    ['naive\_eq', 'naive\_indent'(if possible),'sol2dcf'] en cours de debug
    remarque, au lieu d'ecrie agent(agX) on ecrit agent(X) ne change rien 
    normalement

\setlength{\parskip}{1ex}
    \end{boxedminipage}

    \label{etude:problems:partitioning:naive_eq}
    \index{etude \textit{(package)}!etude.problems \textit{(package)}!etude.problems.partitioning \textit{(module)}!etude.problems.partitioning.naive\_eq \textit{(function)}}

    \vspace{0.5ex}

\hspace{.8\funcindent}\begin{boxedminipage}{\funcwidth}

    \raggedright \textbf{naive\_eq}(\textit{file\_sol}, \textit{nbAgents})

\setlength{\parskip}{2ex}
\setlength{\parskip}{1ex}
    \end{boxedminipage}

    \label{etude:problems:partitioning:create_a_FileSol_wit_a_TPdistribution_for_each_agent}
    \index{etude \textit{(package)}!etude.problems \textit{(package)}!etude.problems.partitioning \textit{(module)}!etude.problems.partitioning.create\_a\_FileSol\_wit\_a\_TPdistribution\_for\_each\_agent \textit{(function)}}

    \vspace{0.5ex}

\hspace{.8\funcindent}\begin{boxedminipage}{\funcwidth}

    \raggedright \textbf{create\_a\_FileSol\_wit\_a\_TPdistribution\_for\_each\_agent}(\textit{file\_dcf}, \textit{percTotal}, \textit{seuilMin}={\tt 1}, \textit{method}={\tt \texttt{'}\texttt{short}\texttt{'}}, \textit{outfilename}={\tt None})

    \vspace{-1.5ex}

    \rule{\textwidth}{0.5\fboxrule}
\setlength{\parskip}{2ex}
    return a new sol with this distribution and a new dcf with a new 
    filename percTotal = [0;100]

    on considere ici que toutes les clauses sont des axioms method = 
    ['random', 'short']

\setlength{\parskip}{1ex}
    \end{boxedminipage}


%%%%%%%%%%%%%%%%%%%%%%%%%%%%%%%%%%%%%%%%%%%%%%%%%%%%%%%%%%%%%%%%%%%%%%%%%%%
%%                               Variables                               %%
%%%%%%%%%%%%%%%%%%%%%%%%%%%%%%%%%%%%%%%%%%%%%%%%%%%%%%%%%%%%%%%%%%%%%%%%%%%

  \subsection{Variables}

    \vspace{-1cm}
\hspace{\varindent}\begin{longtable}{|p{\varnamewidth}|p{\vardescrwidth}|l}
\cline{1-2}
\cline{1-2} \centering \textbf{Name} & \centering \textbf{Description}& \\
\cline{1-2}
\endhead\cline{1-2}\multicolumn{3}{r}{\small\textit{continued on next page}}\\\endfoot\cline{1-2}
\endlastfoot\raggedright \_\-\_\-p\-a\-c\-k\-a\-g\-e\-\_\-\_\- & \raggedright \textbf{Value:} 
{\tt \texttt{'}\texttt{etude.problems}\texttt{'}}&\\
\cline{1-2}
\end{longtable}

    \index{etude \textit{(package)}!etude.problems \textit{(package)}!etude.problems.partitioning \textit{(module)}|)}
