%
% API Documentation for API Documentation
% Package etude
%
% Generated by epydoc 3.0.1
% [Wed Sep  9 11:36:00 2015]
%

%%%%%%%%%%%%%%%%%%%%%%%%%%%%%%%%%%%%%%%%%%%%%%%%%%%%%%%%%%%%%%%%%%%%%%%%%%%
%%                          Module Description                           %%
%%%%%%%%%%%%%%%%%%%%%%%%%%%%%%%%%%%%%%%%%%%%%%%%%%%%%%%%%%%%%%%%%%%%%%%%%%%

    \index{etude \textit{(package)}|(}
\section{Package etude}

    \label{etude}

%%%%%%%%%%%%%%%%%%%%%%%%%%%%%%%%%%%%%%%%%%%%%%%%%%%%%%%%%%%%%%%%%%%%%%%%%%%
%%                                Modules                                %%
%%%%%%%%%%%%%%%%%%%%%%%%%%%%%%%%%%%%%%%%%%%%%%%%%%%%%%%%%%%%%%%%%%%%%%%%%%%

\subsection{Modules}

\begin{itemize}
\setlength{\parskip}{0ex}
\item \textbf{analysis}
  \textit{(Section \ref{etude:analysis}, p.~\pageref{etude:analysis})}

\item \textbf{constants}
  \textit{(Section \ref{etude:constants}, p.~\pageref{etude:constants})}

\item \textbf{exceptions}: to be logged, an exception must not be caught in the same function where it
was raised (at least one level of difference).



  \textit{(Section \ref{etude:exceptions}, p.~\pageref{etude:exceptions})}

\item \textbf{experiments}
  \textit{(Section \ref{etude:experiments}, p.~\pageref{etude:experiments})}

  \begin{itemize}
\setlength{\parskip}{0ex}
    \item \textbf{experiment}
  \textit{(Section \ref{etude:experiments:experiment}, p.~\pageref{etude:experiments:experiment})}

    \item \textbf{singlerunner}: TODO rajouter le logging dedans (avec un paramètre en entrée, \# sinon 
génère automatiquement un fichier singleRun.log et écrit sur le stdout) \# 
réécrire le main ici dans une classe experiment



  \textit{(Section \ref{etude:experiments:singlerunner}, p.~\pageref{etude:experiments:singlerunner})}

  \end{itemize}
\item \textbf{problems}
  \textit{(Section \ref{etude:problems}, p.~\pageref{etude:problems})}

  \begin{itemize}
\setlength{\parskip}{0ex}
    \item \textbf{partitioning}
  \textit{(Section \ref{etude:problems:partitioning}, p.~\pageref{etude:problems:partitioning})}

  \end{itemize}
\item \textbf{utils}
  \textit{(Section \ref{etude:utils}, p.~\pageref{etude:utils})}

  \begin{itemize}
\setlength{\parskip}{0ex}
    \item \textbf{argsGenerator}
  \textit{(Section \ref{etude:utils:argsGenerator}, p.~\pageref{etude:utils:argsGenerator})}

    \item \textbf{futils}
  \textit{(Section \ref{etude:utils:futils}, p.~\pageref{etude:utils:futils})}

    \item \textbf{mylogging}: This class is meant to be inherited, by each module to offer a basic 
interface to the other modules and force the design of a workflow (at least
minimal with \_\_init\_\_())



  \textit{(Section \ref{etude:utils:mylogging}, p.~\pageref{etude:utils:mylogging})}

    \item \textbf{sysutils}
  \textit{(Section \ref{etude:utils:sysutils}, p.~\pageref{etude:utils:sysutils})}

  \end{itemize}
\end{itemize}


%%%%%%%%%%%%%%%%%%%%%%%%%%%%%%%%%%%%%%%%%%%%%%%%%%%%%%%%%%%%%%%%%%%%%%%%%%%
%%                               Variables                               %%
%%%%%%%%%%%%%%%%%%%%%%%%%%%%%%%%%%%%%%%%%%%%%%%%%%%%%%%%%%%%%%%%%%%%%%%%%%%

  \subsection{Variables}

    \vspace{-1cm}
\hspace{\varindent}\begin{longtable}{|p{\varnamewidth}|p{\vardescrwidth}|l}
\cline{1-2}
\cline{1-2} \centering \textbf{Name} & \centering \textbf{Description}& \\
\cline{1-2}
\endhead\cline{1-2}\multicolumn{3}{r}{\small\textit{continued on next page}}\\\endfoot\cline{1-2}
\endlastfoot\raggedright \_\-\_\-p\-a\-c\-k\-a\-g\-e\-\_\-\_\- & \raggedright \textbf{Value:} 
{\tt None}&\\
\cline{1-2}
\end{longtable}

    \index{etude \textit{(package)}|)}
